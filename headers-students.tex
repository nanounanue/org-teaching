% This defines the LaTeX headers that will be included in the
% generated LaTeX source for the handbook, in order to define a
% handbook title page

% This is a hack as I'm no LaTeX guru, and I just borrowed stuff from
% colleagues. Clearly could be improved. Contributions welcome.

% \usepackage[french]{babel}
\usepackage[top=20mm, bottom=20mm, left=25mm , right=25mm]{geometry}

%% \usepackage{amssymb,amsmath}
\usepackage{fancyhdr} %For headers and footers
\pagestyle{fancy} %For headers and footers
\usepackage{lastpage} %For getting page x of y
%% \usepackage{float} %Allows the figures to be positioned and formatted nicely
%% \floatstyle{boxed} %using this
%% \restylefloat{figure} %and this command
%% \usepackage{url} %Formatting of yrls
\lhead{Org Teaching}
\chead{}
\rhead{}
\lfoot{Student handbook}
\cfoot{}
\rfoot{\thepage}


\usepackage{framed}
\usepackage{xcolor}
\definecolor{shadecolor}{gray}{.95}
\newenvironment{NOTES}{\begin{lrbox}{\mybox}\begin{minipage}{0.9\textwidth}\begin{shaded}}{\end{shaded}\end{minipage}\end{lrbox}\fbox{\usebox{\mybox}}}

%%\documentclass[10pt]{article}

%%\usepackage[utf8]{inputenc}
%%\usepackage[T1]{fontenc} % caractères accentués en entrée, dans emacs
%%\usepackage[francais,frenchb,english]{babel}
%%\selectlanguage{frenchb}
%%\FrenchFootnotes

%%\usepackage{graphicx}

%%\begin{document}


\newcommand*{\mycustomhandbooktitle}[4]{%
  \begin{titlepage}
    \null\vspace*{\stretch{1}}
    \rule{\linewidth}{1mm}
    \begin{center}
      {\Huge \textsc{#1} \par\vskip 1em}
      % affiche soit une image (mainlogo) soit le propriologo
      {\Huge \includegraphics[scale=1.0]{media/your_logo} \par\vskip 1em}
      {\Large \textsc{#2} \par\vskip 1em}
      {\Large\textsc{} \par\vskip 1em}
      {\small #3 \par\vskip 1em}
      {\huge \textsc{#4} \par\vskip 1em}
    \end{center}
    \rule{\linewidth}{1mm}
    \vfill
  \end{titlepage}
}

% Use Verdana, since we compile LaTeX with lualatex
\usepackage{fontspec}
\defaultfontfeatures{Ligatures=TeX} % to have the automatics ligatures of TeX
\setromanfont{Verdana}
% don't justify (better for dyslexic readers ?)
\usepackage[document]{ragged2e}
