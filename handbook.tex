% Created 2016-12-20 mar. 16:19
% Intended LaTeX compiler: pdflatex
\documentclass[a4paper]{article}
\usepackage[utf8]{inputenc}
\usepackage[T1]{fontenc}
\usepackage{graphicx}
\usepackage{grffile}
\usepackage{longtable}
\usepackage{wrapfig}
\usepackage{rotating}
\usepackage[normalem]{ulem}
\usepackage{amsmath}
\usepackage{textcomp}
\usepackage{amssymb}
\usepackage{capt-of}
\usepackage{hyperref}
% This defines the LaTeX headers that will be included in the
% generated LaTeX source for the handbook, in order to define a
% handbook title page

% This is a hack as I'm no LaTeX guru, and I just borrowed stuff from
% colleagues. Clearly could be improved. Contributions welcome.

% \usepackage[french]{babel}
\usepackage[top=20mm, bottom=20mm, left=25mm , right=25mm]{geometry}

%% \usepackage{amssymb,amsmath}
\usepackage{fancyhdr} %For headers and footers
\pagestyle{fancy} %For headers and footers
\usepackage{lastpage} %For getting page x of y
%% \usepackage{float} %Allows the figures to be positioned and formatted nicely
%% \floatstyle{boxed} %using this
%% \restylefloat{figure} %and this command
%% \usepackage{url} %Formatting of yrls
\lhead{Org Teaching}
\chead{}
\rhead{}
\lfoot{Student handbook}
\cfoot{}
\rfoot{\thepage}


\usepackage{framed}
\usepackage{xcolor}
\definecolor{shadecolor}{gray}{.95}
\newenvironment{NOTES}{\begin{lrbox}{\mybox}\begin{minipage}{0.9\textwidth}\begin{shaded}}{\end{shaded}\end{minipage}\end{lrbox}\fbox{\usebox{\mybox}}}

%%\documentclass[10pt]{article}

%%\usepackage[utf8]{inputenc}
%%\usepackage[T1]{fontenc} % caractères accentués en entrée, dans emacs
%%\usepackage[francais,frenchb,english]{babel}
%%\selectlanguage{frenchb}
%%\FrenchFootnotes

%%\usepackage{graphicx}

%%\begin{document}


\newcommand*{\mycustomhandbooktitle}[4]{%
  \begin{titlepage}
    \null\vspace*{\stretch{1}}
    \rule{\linewidth}{1mm}
    \begin{center}
      {\Huge \textsc{#1} \par\vskip 1em}
      % affiche soit une image (mainlogo) soit le propriologo
      {\Huge \includegraphics[scale=1.0]{media/your_logo} \par\vskip 1em}
      {\Large \textsc{#2} \par\vskip 1em}
      {\Large\textsc{} \par\vskip 1em}
      {\small #3 \par\vskip 1em}
      {\huge \textsc{#4} \par\vskip 1em}
    \end{center}
    \rule{\linewidth}{1mm}
    \vfill
  \end{titlepage}
}


\rhead{Example lesson}
\author{Olivier Berger}
\date{2016-12-19}
\title{How to teach using org-mode for fun and profit}
\hypersetup{
 pdfauthor={Olivier Berger},
 pdftitle={How to teach using org-mode for fun and profit},
 pdfkeywords={},
 pdfsubject={Olivier Berger's org-mode framework for teaching},
 pdfcreator={Emacs 24.5.1 (Org mode 9.0)}, 
 pdflang={English}}
\begin{document}

\mycustomhandbooktitle{How to teach using org-mode for fun and profit}{Olivier Berger}{(2016-12-19)}{org-teaching - example lesson}
\setcounter{tocdepth}{1}
\tableofcontents




\section{Introduction}
\label{sec:org18555cc}

This is a demo document about the
\href{http://www-public.tem-tsp.eu/~berger_o/org-teaching/}{codename \texttt{org-teaching}} framework.

\begin{itemize}
\item \emph{You're viewing a \href{http://lab.hakim.se/reveal-js/}{reveal.js} Web slides deck. You may press 's' to view presenter notes.}
\end{itemize}

\begin{NOTES}
You'll find the source code at Gitlab : \url{https://gitlab.com/olberger/org-teaching}

I mention \emph{codename} \texttt{org-teaching} because it really hasn't yet a proper name, and may never have. This is far from a full fledged product.
\end{NOTES}

\subsection{Org-mode powa}
\label{sec:org0962657}

This framework heavily relies on \href{http://orgmode.org/}{org-mode} (version 9 at the time of writing) and the \href{https://github.com/yjwen/org-reveal/}{org-reveal} exporter for \texttt{reveal.js}.

\section{About this handbook}
\label{sec:org145e73a}

This PDF handbook is one variant of the same teaching material, also available \href{./slides.html}{as a slides deck}.

This section may contain content that is best viewed in (\LaTeX{}) PDF export of org-mode.

\section{Features}
\label{sec:org2d01d1d}
\subsection{Writing teaching material in org-mode}
\label{sec:org33c98d7}

The goal is to be able to edit a single file (namely \texttt{lesson.org}) which will contain, in a single source, all the content of a lesson, written with org-mode syntax.

From this single source, several documents are generated :
\begin{itemize}
\item \textbf{slides} (as a dynamic Web document) for overhead presentation
\item a \textbf{handbook} that contains the same information (or more) and can be handed to the students for work outside the classroom (a PDF file)
\item another \textbf{handbook} for the teaching team, to provide additional instructions (also a PDF file)
\end{itemize}

\begin{NOTES}
The handbook \LaTeX{} formatting has a summarized content table and nice looking (in principle) title page. Customize at will

The teachers handbook contains a less nice-looking format, includes a detailed table of contents, and has a watermark. Again, customize at will and submit improvements.
\end{NOTES}

\subsection{org-reveal slides}
\label{sec:org9a3c9fb}

Pretty much all features of \texttt{reveal.js} supported by the org-mode reveal.js exporter \href{https://github.com/yjwen/org-reveal/}{org-reveal}, should be supported too.

An example \texttt{org-reveal} document is \href{./elisp/org-reveal/Readme.html}{available here} for inspiration (it's the export of org-reveal's \texttt{README.org}, actually).

\begin{NOTES}
If you're already familiar with reveal.js, you may have noticed that the current settings adopted for our slides generation are quite frugal: no fancy 3D effects and likes.

It's a matter of taste : I didn't want to show off, and prefer to give students a clear content on the projector behind me.
\end{NOTES}

\subsection{Structure of the sections / slides}
\label{sec:org8cda526}

I'm using the 3 levels of outlining / sectioning so that the content can be sectioned in the same way in \texttt{lesson.org} and appear appropriately in the slides and handbook, with these principles:

\begin{enumerate}
\item First level outlines define main sections of the document.
\item Second level outlines are the main "horizontal" slides that will be played with page up/down
\item Third level outlines may be used for additional content ("vertical" slides) that may be skipped for the presentation, but is still accessible with cursor keys.
\end{enumerate}

\begin{NOTES}
The first level outlines can be rendered as a "separating" slides which may get a different \texttt{reveal\_background} and \texttt{class="center"} slide layout, but that isn't automatic. See \hyperref[sec:orgff9c296]{Section separators}.
\end{NOTES}

\subsection{Presenter notes / content for the handbook}
\label{sec:org0f7fb15}

\href{https://github.com/yjwen/org-reveal/#speaker-notes}{org-reveal's \emph{Speaker notes}} may be added to the slides (and will only appear on dual-screen presentation after having pressed 's': standard reveal.js feature).

They will be masked for the audience, but will, by default appear in the handbook given to the students.

\begin{verbatim}
#+BEGIN_NOTES
This is a note
#+END_NOTES
\end{verbatim}

\subsection{Masking content for some audiences}
\label{sec:orge883b3d}

I've implemented some easy ways to preserve some of the content of the same \texttt{lesson.org} source for certain outputs (using org exporter's standard \texttt{EXCLUDE\_TAGS}):

\begin{description}
\item[{\emph{Slides only} material}] that won't be embedded in the handbook : surprise stuff for live audience, or HTML-only hacks;
\item[{\emph{Teachers only} material}] secret knowledge that only adults need to know (for instance), which won't be exported;
\item[{\emph{Handbook only} material}] stuff that only fits in the handbook, and/or only exports as \LaTeX{} and not HTML.
\end{description}

\begin{NOTES}
The choice to reveal or not some details to the students is quite arbitrary and depends on your pedagogical approach. I'm not advisable in this matter. YMMV.
\end{NOTES}

\subsection{Stuff only meant for presentation}
\label{sec:orga627dfc}

Tagging a section/slide with \texttt{:slidesonly:} means it isn't exported in the handbooks.

Below is an example (or not)\ldots{}

\subsubsection{Regular slide (no tag on heading line)}
\label{sec:orga15f652}

There should be no "Only in the slides" after this section, in the
handbooks, as it has been tagged with \texttt{slidesonly}.

\subsection{Stuff only meant for teachers}
\label{sec:org8b36cea}

Tagging a section/slide with \texttt{:teachersonly:} means it isn't exported in the students handbook (nor in the slides).

Below is an example\ldots{}

\subsubsection{Regular slide (no tag on heading line)}
\label{sec:org46bb8f7}

There should be no "Only for teachers" after this section, in the slides or in the
students handbook, as it has been tagged with \texttt{teachersonly}.

\subsection{Notes only for the teachers}
\label{sec:orgf6aa41d}

This slide/section contains notes, but only part of it is displayed in
the presentation notes included in the handbook. Special notes and are
kept only for the teachers handbook.

We use an org-mode drawer for that (additional benefit is that the content is folded by default in emacs, as it may be verbose and/or "sensitive") :
\begin{verbatim}
#+BEGIN_NOTES
This part of the note can be viewed by the students in the handbook.

:TEACHERSONLY:
Not this one
:END:
#+END_NOTES
\end{verbatim}

\begin{NOTES}


This part of the note can be viewed by the students in the handbook,
but not the rest.
\end{NOTES}

\subsection{Stuff only in the handbooks}
\label{sec:org1cbc2c2}

Just like sections are for slides only, others can be for the handbook
only, using the \texttt{handbookonly} tag. This may be useful for \textbf{Annex}
sections for instance, or for stuff that the HTML exporter won't like, with inline \LaTeX{}.

\subsection{Code colorization}
\label{sec:org750a11d}
Code is colorized / highlighted in the slides :-)

\begin{NOTES}
Nice when like me, you're teaching Computer Science stuff
\end{NOTES}
\subsection{Misc org-mode}
\label{sec:orgd85ffef}

\subsubsection{Babel powa}
\label{sec:orgef8e7bf}
As you're using org-mode, its \texttt{babel} components are available, to embed source code in the same \texttt{lesson.org} source, and manage executable code and teaching material at once.

Look for \emph{literate programing} instructions in the \href{http://orgmode.org/manual/Working-with-source-code.html}{org-mode docs} to know more.

\subsection{Missing features ?}
\label{sec:org35c3c46}

Please try and talk to me to suggest new stuff and/or provide patches ;)

\begin{NOTES}
See the teacher's handbook for some ideas
\end{NOTES}

\section{HowTo}
\label{sec:org5ef9246}
\subsection{Use Emacs}
\label{sec:orgaba3df7}

There's no production chain, makefiles or other authoring tools. 

No new Elisp either.

\begin{NOTES}
You're welcome to suggest improvements. But I'm not an Elisp hacker,
so I may not be able to maintain them. At the moment, the intent is to
rely on the original org-reveal only, as much as possible.
\end{NOTES}

\subsection{Modify only the lesson.org}
\label{sec:orgcc7537d}

\textbf{Only one file should be edited for writing the lesson's material : \texttt{lesson.org}}

Only exception is modification of some configurations for title pages
and other bits that shouldn't change much in time (see next section).

\subsection{Configuration}
\label{sec:org61dbc8b}

Each lesson.org needs some configuration :
\begin{itemize}
\item Configure \texttt{org-reveal-title-slide} in \texttt{slides.org}.

\item Configure header (\texttt{\textbackslash{}lhead\{...\}} and \texttt{\textbackslash{}rhead\{...\}}) and footer
(\texttt{\textbackslash{}lfoot\{...\}} and \texttt{\textbackslash{}rfoot\{...\}}) elements in the headers like
\texttt{\#+LaTeX\_HEADER: \textbackslash{}rhead\{...\}} in \texttt{handbook.org} and
\texttt{teacher-handbook.org}.
\end{itemize}

\begin{NOTES}
These may be better handled, but some limitations of the exporters or
my lack of knowledge/time have prevented a better result so
far. Improvements much welcome.
\end{NOTES}

\subsection{Generating final documents}
\label{sec:org7701118}

We're using the standard exporters so each output format will be exported from its corresponding umbrella \texttt{.org} source :

\begin{description}
\item[{slides}] open \texttt{slides.org}, then \texttt{C-c C-e R ...} for \texttt{org-reveal} export (to \texttt{slides.html}), provided that you have loaded org-reveal in Emacs
\item[{handbook}] open \texttt{handbook.org}, then \texttt{C-c C-e l ...} for \LaTeX{} export (to \texttt{handbook.pdf})
\item[{teacher handbook}] open \texttt{teacher-handbook.org}, then \texttt{C-c C-e l ...} for \LaTeX{} export (to \texttt{teacher-handbook.pdf})
\end{description}

\subsubsection{Exporting slides to HTML with org-reveal}
\label{sec:org7f817f4}

Depending on how you installed org-reveal (\hyperref[sec:org63fbe85]{Git submodules} or otherwise), \texttt{org-reveal} may already be available.

If not yet, load it with \texttt{M-x load-file} from the location of its Git submodule (\texttt{elisp/org-reveal/ox-reveal.el} by default).

\begin{NOTES}
I'm not sure which solution is better : org-reveal from Git (hence the Git submodule) or from an Emacs package. Please report.
\end{NOTES}

\subsubsection{Printing slides}
\label{sec:org6d24b33}

TBD

\subsection{Known Issues}
\label{sec:org777cc39}

\subsubsection{Firefox issues ?}
\label{sec:orgdbf2b3a}

We have experienced issues with presentations made on some versions of Firefox, which are known by reveal.js maintainer\ldots{} maybe best viewed in chrome.

You may prefer to have a PDF variant of the slides (see \hyperref[sec:org6d24b33]{Printing slides}) in case.



\section{How it works / Installation}
\label{sec:orgcc1cf8a}
\subsection{Use the source Luke}
\label{sec:org6264da9}

See the contents of the files\ldots{} but be wary that it's sometimes messy and incrementally obtained.

Emacs is your buddy.

Git clone from \texttt{https://gitlab.com/olberger/org-teaching.git} (see the \href{https://gitlab.com/olberger/org-teaching}{Gitlab project})

\subsection{Git submodules}
\label{sec:org63fbe85}

The repository contains Git submodules for :
\begin{itemize}
\item reveal.js
\item org-reveal
\end{itemize}

So :
\begin{verbatim}
git submodule init
git submodule update
\end{verbatim}
You may prefer to install them another way (ELPA repo, CDN, etc.)

\begin{NOTES}
Refer to \href{https://github.com/yjwen/org-reveal/#requirements-and-installation}{org-reveal's documentation} for more details.
\end{NOTES}

\subsection{Slides appearance}
\label{sec:org02c65d9}

\subsubsection{Reveal.js settings}
\label{sec:orge76d4e8}

See the org-reveal settings set in the sources and the docs for a detailed explanation.

I'm using the following for a "frugal" look close to what
powerpoint or beamer (?) could look like :

\begin{verbatim}
#+REVEAL_HLEVEL: 2
#+REVEAL_THEME: simple
#+REVEAL_TRANS: fade
#+REVEAL_SPEED: fast
#+REVEAL_MARGIN: 0.0
#+REVEAL_EXTRA_CSS: ./presentation.css
#+REVEAL_ROOT: ./reveal.js

#+OPTIONS: reveal_center:nil
\end{verbatim}

\subsubsection{Section separators}
\label{sec:orgff9c296}

The highest level sections include the following properties below the heading line, to customize the look of the slide. 

\begin{verbatim}
:PROPERTIES:
:REVEAL_EXTRA_ATTR: class="center"
:reveal_background: #dbdbed
:END:
\end{verbatim}

This is intended to provide some visual sense of the transitions between sections. Please adapt and report.

\subsubsection{Title screen picture (logos, etc.)}
\label{sec:org5c188f5}

I'm not yet sure how much may be achieved with HTML and CSS for the
title page of the slides deck, so I've relied on the embedding of a
background image that will contain the logos and additional graphics. 

\begin{verbatim}
#+REVEAL_TITLE_SLIDE_BACKGROUND: ./media/title-slide-background.png
\end{verbatim}

I'm quite sure this could be improved.


\section{Annex}
\label{sec:orgb94cd6b}

\subsection{Thanks}
\label{sec:org7290bc6}

\begin{itemize}
\item All contributors to org-mode (special kudos to Carsten Dominik and Bastien Guerry)
\item Yujie Wen for \texttt{org-reveal}
\item Hakim El Hattab for \texttt{reveal.js}
\item My colleagues at Telecom SudParis who had to teach with this tool without much rehersal
\item Our students who endured it for a brand new course (and included bugs)
\end{itemize}

\subsection{Feedback}
\label{sec:orgece67ef}

I may be contacted from \href{http://www-public.tem-tsp.eu/~berger_o/#sec-3}{my Web page} or via \href{https://gitlab.com/olberger/org-teaching}{the Gitlab project}.

\subsection{Usage reports}
\label{sec:orgbcb0f65}

\subsubsection{2016 at Telecom SudParis}
\label{sec:org1110680}

Created and used for the first time for teaching "Web Architecture and Applications" in the CSC4101 module at Telecom SudParis (Olivier Berger and colleagues)
\end{document}
